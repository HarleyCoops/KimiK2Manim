\documentclass[tikz,border=10pt]{standalone}
\usepackage{tikz}
\usetikzlibrary{shapes,arrows,positioning,fit,backgrounds,decorations.pathreplacing,calc}

\begin{document}
\begin{tikzpicture}[
    % Node styles
    agent/.style={rectangle, rounded corners, minimum width=4cm, minimum height=2cm, 
                  text centered, draw=black, fill=blue!25, font=\small\bfseries, align=center},
    data/.style={rectangle, rounded corners, minimum width=4.5cm, minimum height=2.5cm,
                 text centered, draw=black, fill=green!20, font=\tiny, align=left},
    output/.style={rectangle, rounded corners, minimum width=5cm, minimum height=3cm,
                   text centered, draw=black, fill=orange!25, font=\small, align=left},
    arrow/.style={->, >=stealth, thick, color=black},
    label/.style={font=\tiny, color=darkgray, align=center}
]

% Input
\node[agent, fill=purple!25] (input) at (0,8) {
    \textbf{User Prompt}\\
    \textit{"explain pythagorean theorem"}
};

% Stage 1: Prerequisite Explorer
\node[agent, fill=blue!30] (explorer) at (0,6) {
    \textbf{1. KimiPrerequisiteExplorer}\\
    \textit{Recursively builds knowledge tree}
};

% Tree structure output
\node[data, fill=green!15] (tree1) at (-4,4) {
    \textbf{KnowledgeNode Tree}\\
    \vspace{0.1cm}
    \textit{Structure:}\\
    • concept: "pythagorean theorem"\\
    • depth: 0\\
    • is\_foundation: true\\
    • prerequisites: []\\
    \vspace{0.1cm}
    \textit{Recursive: explores prerequisites}
};

% Stage 2: Mathematical Enricher
\node[agent, fill=green!30] (math) at (-4,1.5) {
    \textbf{2. KimiMathematicalEnricher}\\
    \textit{Enriches each node with math}
};

% Math-enriched tree
\node[data, fill=green!20] (tree2) at (-4,-1) {
    \textbf{Math-Enriched Tree}\\
    \vspace{0.1cm}
    \textit{Added to each node:}\\
    • equations: ["a²+b²=c²", ...]\\
    • definitions: \{a: "...", b: "..."\}\\
    • interpretation: "..."\\
    • examples: ["3-4-5 triangle", ...]\\
    • typical\_values: \{"3-4-5": "..."\}
};

% Stage 3: Visual Designer
\node[agent, fill=yellow!30] (visual) at (0,1.5) {
    \textbf{3. KimiVisualDesigner}\\
    \textit{Designs visual specifications}
};

% Visual-enriched tree
\node[data, fill=yellow!20] (tree3) at (0,-1) {
    \textbf{Visual-Enriched Tree}\\
    \vspace{0.1cm}
    \textit{Added visual\_spec:}\\
    • visual\_description: "right triangle..."\\
    • color\_scheme: "blue, green, red"\\
    • animation\_description: "draws itself..."\\
    • transitions: "fade in..."\\
    • camera\_movement: "zoom in..."\\
    • duration: 15\\
    • layout: "center triangle..."
};

% Stage 4: Narrative Composer
\node[agent, fill=orange!30] (narrative) at (4,1.5) {
    \textbf{4. KimiNarrativeComposer}\\
    \textit{Creates verbose narrative}
};

% Final enriched tree
\node[data, fill=orange!20] (tree4) at (4,-1) {
    \textbf{Final Enriched Tree}\\
    \vspace{0.1cm}
    \textit{Added narrative:}\\
    • verbose\_prompt: "2000+ words"\\
    • concept\_order: ["pythagorean theorem"]\\
    • total\_duration: 15\\
    • scene\_count: 1\\
    \vspace{0.1cm}
    \textit{Complete prompt with:}\\
    • LaTeX equations\\
    • Visual descriptions\\
    • Animation timing\\
    • Color schemes\\
    • Transitions
};

% Final Output: Manim Code
\node[output, fill=red!25] (manim) at (0,-4) {
    \textbf{Fully Rendered Manim Animation}\\
    \vspace{0.2cm}
    \textit{Complete Python code:}\\
    • Scene class definition\\
    • Visual objects (Triangle, Square, MathTex)\\
    • Animations (Create, FadeIn, Transform)\\
    • LaTeX equations rendered\\
    • Color schemes applied\\
    • Timing and transitions\\
    • Camera movements\\
    \vspace{0.2cm}
    \textit{Ready to render video}
};

% Flow arrows
\draw[arrow, thick] (input) -- (explorer) node[midway, right, label] {concept};
\draw[arrow, thick] (explorer) -- (tree1) node[midway, left, label] {recursive\\exploration};
\draw[arrow, thick] (tree1) -- (math) node[midway, left, label] {KnowledgeNode};
\draw[arrow, thick] (math) -- (tree2) node[midway, left, label] {math\\enriched};
\draw[arrow, thick] (tree2) -- (visual) node[midway, below, label, sloped] {math +\\structure};
\draw[arrow, thick] (explorer) -- (visual) node[midway, right, label] {KnowledgeNode};
\draw[arrow, thick] (visual) -- (tree3) node[midway, right, label] {visual\\specs};
\draw[arrow, thick] (tree3) -- (narrative) node[midway, below, label, sloped] {complete\\tree};
\draw[arrow, thick] (tree2) -- (narrative) node[midway, below, label, sloped] {math +\\visuals};
\draw[arrow, thick] (narrative) -- (tree4) node[midway, right, label] {verbose\\prompt};
\draw[arrow, thick, red] (tree4) -- (manim) node[midway, right, label] {\textbf{2000+ word\\narrative}};
\draw[arrow, thick, red] (tree3) -- (manim) node[midway, left, label, sloped] {visual\\specs};

% Background stages
\begin{scope}[on background layer]
    \node[fill=blue!5, rounded corners, fit=(explorer) (tree1), inner sep=5pt, 
          label={[font=\small\bfseries, anchor=north west]above left:Stage 1: Tree Building}] {};
    \node[fill=green!5, rounded corners, fit=(math) (tree2), inner sep=5pt,
          label={[font=\small\bfseries, anchor=north west]above left:Stage 2: Math Enrichment}] {};
    \node[fill=yellow!5, rounded corners, fit=(visual) (tree3), inner sep=5pt,
          label={[font=\small\bfseries, anchor=north west]above left:Stage 3: Visual Design}] {};
    \node[fill=orange!5, rounded corners, fit=(narrative) (tree4), inner sep=5pt,
          label={[font=\small\bfseries, anchor=north west]above left:Stage 4: Narrative}] {};
    \node[fill=red!5, rounded corners, fit=(manim), inner sep=5pt,
          label={[font=\small\bfseries, anchor=north west]above left:Final Output: Manim Code}] {};
\end{scope}

% Title
\node[above=0.5cm of input, font=\Large\bfseries] (title) {
    KimiK2Manim Pipeline: Agent Flow and Content Evolution
};

% Legend
\node[below=0.3cm of manim, font=\tiny, align=center, text width=12cm] (legend) {
    \textbf{Pipeline Characteristics:}\\
    • \textit{Recursive Processing:} Each agent processes the entire tree recursively\\
    • \textit{Incremental Enrichment:} Each stage adds a new layer of detail\\
    • \textit{Cumulative Information:} Later stages use all previous enrichments\\
    • \textit{Final Output:} Complete narrative contains everything needed for Manim generation
};

\end{tikzpicture}
\end{document}

